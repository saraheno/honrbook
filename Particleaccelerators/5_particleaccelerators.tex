\section{Introduction}

\noindent
It's  a nice sunny day and you went for a picnic up a nice little mountain in Switzerland. You enjoyed your day and it t until you started your journey back when you noticed that the gas tank is almost empty. There are no gas stations around and that makes you worried until your friend says ``Never mind, we will use the force of gravity to accelerate us down''. Ahha! - you, being an avid student of physics, know exactl what your friend is talking about. Don't you?

\;
\noindent
In our everyday experience we know that objects are accelerated when moved by some force from one point to the other. This can be gravitational force or electromagnetic force.
We also know that higher the speed of an object, the more smashing energy it has. The damage to car moving with 5 mph in the parking lot is much less compared to the damage to a car on the highway when moving with a speed of 100 mph.

\;
\noindent
These two everyday observations are basic principles of particle accelerators - machines that accelerate very small particles to very high speeds and ''damage'' or break them either by smashing them into one another (like two cars head on) or on to a target (like a car colliding a wall). The difference is that the tiny paricles like protons, can be accelerated to very high speeds compared to a car on the highway. The Large Hadron Collider, the accelerator that discovered the Higgs boson just a couple years ago, can accelerate proton almost to the speed of light (v = 0.999999991*c = 299792455 m/s = 186282 miles/sec). The smashing power or energy of such particles on collision will not only break them, but, according to Einstein's famous E=mc2, part of this energy will actually turn into new particles. Imagine collision of two cars at a very high speed and not only the car parts but also a few doves, a helicopter and a school bus comes flying out.

\;
\noindent
Of course, accelerating tiny particles like protons is very different from accelerating a car. That's why weneedvery complicated and big machines like the LHC. Many times, the technology needed to meet the demand ofthe physical process we are trying to make happen and observe is not available. In these cases physicists andengineers work together and push the boundaries of technology.

\;

In general, there are three types of particle accelerators: 

\;

           \cdot \:  Circular Accelerators

	   \cdot \:  Linear Accelerators (LINAC)

\;
\noindent
A particle accelerator could be a combination of these general forms. For example the large hadron collider not only makes use of both circular and linear accelerators (Fig. 1).

\section{Parts of an Accelerator}

Whether linear or circular, the basic parts of an accelerator are the same and are described below.

\;


\noindent
\textbf{Beam of particles for collisions}

\;
\;

\noindent
These are the particles that will be accelerated and then collided. LHC accelerates protons and heavy ions such as lead. For the LHC beam, 300 trillion protons are required, but since a single cubic centimetre of hydrogen gas at room temperature contains about 60 million trillion protons, the LHC can be refilled 200 000 times with just one cubic centimetre of gas - and it only needs refilling twice a day!

\;
\noindent
These protons are supplied from a hydrogen gas bottle. Hydrogen atoms consist of a proton and an electron. After stripping the hydrogen atom of its only electron we are left with a proton, which is then accelerated to required energy before colliding with protons accelerated in the opposite direction.

\;

\noindent
\textbf{Beam Pipe}

\\
\noindent 
This is a metal pipe inside which the beam of particles travels. For the LHC, we have two beam pipes for opposite traveling beams of particles. This pipe has to be empty of any other atoms (e.q., the one in air) to avoid collisions between gas molecules and the particles in the beam (Why is this a bad thing?). The pressure inside of the LHC is 10^{13} atm, ten times less then the pressure on The Moon.

\;

\noindent
\textbf{Devices to change particle speed (Radiofrequency (RF) cavitires and electric fields)}

\;
\noindent
A radiofrequency (RF) cavity is a metallic cavity like structure that looks like beads around the beam pipe.  These cavities contain an electromagnetic field such that when charged particles pass through that field it transfers energy to these particles and they are pushed or accelerated forward along the accelerator. Think of a proton as a surfer riding a wave.  Every electromagnetic wave accelerates a bunch of particles, about 100 billion of them and each of the two beams consists of a number of such bunches, a few meters apart. These bunches are circulated in the beam pipe going around the LHC ring thousands of times per second. It takes about 20 minutes to get to the energies required and during that time the protons cover a distance further than from Earth to the Sun and back.

\;
\noindent
At full power, trillions of protons will race around the LHC accelerator ring 11,245 times a second, travelling at 99.99\% the speed of light. A motionless proton has a mass of 0.938 GeV (938 million electron volts).The accelerators bring them to a final mass (or energy, which in this case is practically the same thing) of 7000 billion electron volts (7 tera-eV or 7 TeV). If you could - hypothetically - accelerate a person of 100 kg in the LHC, his or her mass would end up being 700 t.