\section{Cross section}\index{Cross Section}
Imagine a 2-D box in deep space, away from gravity, containing some number of large 2D disks.  If you threw a ball at the box, the probability of hitting a disks is related to the ratio of the cross sectional area of the box to the cross sectional area of the disks.  The cross sectional area of the disks is their �cross section� and it has units of area.  If you look at a solid and imagine scattering particles off the nuclei inside them, it is similar.  Remember that solids are, after all, mostly empty space.  If you project all the nuclei into the 2D front face of the solid, you have our 2-D box.  The cross section area of a nuclei is measured in fm2.  For low energy scattering by the strong force, since the strong force only has a very short range, and interactions can only happen if the ball virtually touches the nucleus, this is a good approximation.  A unit commonly used in the barn, which is 10-28 m2.  Nuclear �cross sections� are around 80 mbarns.  For forces with a longer range, such as E&M, we define instead an �effective� area.  See any introductory book on particle physics for a more precise definition.

What if instead you threw a steady stream of balls at the box and wanted to calculate the rate at which balls hit beads and are deflected?  Also, as you can imagine, in realistic beams the projectiles do not march single file.  You should imagine them moving in a cylinder with some cross section, like:

If the number of balls in the beam per unit area is na, and the velocity of the balls is v,  then the number of balls reaching the target per second is

\[\Phi = n_{a}v\]

and \(\Phi\) is called the flux.  What are the units of \(\Phi\)?

The number of scatters per second is then

\[\frac{dN}{dt}=\Phi !I could not figure out how to this simble here here!\]

(check that the units work).

Now, you may ask what happens in a collider?  Well, it is just the same.  You can always move to a frame where one of the protons is at rest, do the calculation there, and move back.  We will see in the next section why our calculation of the cross section does not depend on the frame in which it is calculated.

