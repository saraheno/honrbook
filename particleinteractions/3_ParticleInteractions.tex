
\section{Introduction}\index{Introduction}

The purpose of a particle physics detector is find all the particles produced
in a collision, identify which type of particle each is, and measure its 4-momenta.
At the LHC, in a single collision, thousands of particles can be produced.
The particle physics detector must make signals that somehow can yield this information.
To understand how this can happen, we first must understand how particles interact
with matter, so we can learn of the possible types of signals particles could produce.

What kinds of particles can an LHC detector detect?  First, the particle must live long 
enough to reach the detector, without decaying.  The beam pipe of the LHC has a 
radius with dimensions measured in cm.  The particle must exist the beam pipe to
reach the detector and make a signal.  If we take the diameter to be 1 cm, then
the particle life time must be large enough for it to exit.

The particle must also interact strongly enough with matter to produce a signal in a detector that is small enough to be affordable.  While specialized very large detectors exist which can detect neutrinos, this is not possible at a collider detector.

There are only a few types of particles (and their antiparticles) that satisfy these requirements, and they are:
\begin{itemize}
\item leptons
\begin{itemize}
\item electrons
\item muons
\end{itemize}
\item bosons
\begin{itemize}
\item photons
\end{itemize}
\item mesons
\begin{itemize}
\item charged pions
\item k-longs (a very rare particle)
\end{itemize}
\item baryons
\begin{itemize}
\item protons
\item neutrons
\end{itemize}
\end{itemize}






Particles primarily interact with the matter of the detector via electric force, as it is long range (remember that solids are mostly empty space).  The strong force will also play a role, but we will ignore this for now and come back to it when we discuss calorimetry.  The type of interaction, however, varies with the energy of the particle.  We can sometimes think of the electric force as being composed of photons.  Higher energy particles can emit higher energy photons.  The wave length of the photon, as we discussed earlier, is related to the photons energy.  The interactions of those photons with bulk matter will be different if this wave length is short compared to the spacing of the atoms or long compared to this spacing.


\section{Interactions of charged particles with matter}\index{chargedinteractions}

\section{Interactions of gamma rays with matter}\index{gammainteractions}

Here we call them “gamma rays” instead of photons because we are going to discuss only those photons of interest to particle physics: ones with energies above a kev or so.



